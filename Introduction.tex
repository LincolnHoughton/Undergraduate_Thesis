\section{Introduction}\label{Sect:intro}
\par New metal alloys are constantly being developed and implemented in science and industry. The difficulty is in producing a useful alloy and determining its properties. Rather than manufacturing every conceivable alloy in a laboratory, each alloy's properties can be determined computationally. Complex quantum models can be used to generate atomic configuration energies and eventually detailed phase diagrams. A phase diagram is a chart used to show the distinct phases (solid, liquid, gaseous, etc.) of a material for a given temperature and pressure. These alloys can then be evaluated and, if found to be superior, fabricated for use in aircrafts, bridges, batteries, etc. 
\par These quantum models come at a high computational cost, often making data collection a long and drawn out process. It would be useful to construct a simpler model that could be quickly trained and used to predict the atomic configuration energies of a given alloy. Such a model would help alleviate the stress on the main bottleneck to finding novelty alloys, computational power. 
\par The purpose of this research was to investigate the credibility of this concept by attempting to produce a model of this quality. Sections \ref{Sect:background} and \ref{Sect:modelPrep} will cover the math and modeling concepts required to understand the actual model construction in Section \ref{Sect:procedure} and the analysis of its results in Section \ref{Sect:results}. 