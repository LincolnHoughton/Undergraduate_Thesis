\documentclass[oneside,senior]{BYUPhys}
% The BYUPhys class is for producing theses and dissertations
% in the BYU Department of Physics and Astronomy.  You can supply
% the following optional arguments in the square brackets to
% specify the thesis type:
%
%   senior  : Produces the senior thesis preliminary pages (default)
%   honors  : Produces the honors thesis preliminary pages
%   masters : Produces the masters thesis preliminary pages
%   phd     : Produces the PhD dissertation preliminary pages
%
% The default format is appropriate for printing, with blank pages
% inserted after the preliminary pages in twoside mode so you can
% send it directly to a two-sided printer. However, for ETD
% submission the blank pages need to be removed from the final output.
% The following option does this for you:
%
%   etd     : Produces a copy with no blank pages in the preliminary section.
%             Remove this option to produce a version with blank pages inserted
%             for easy double sided printing.
%
% The rest of the class options are the same as the regular book class.
% A few to remember:
%
%   oneside : Produces single sided print layout (recommended for theses less than 50 pages)
%   twoside : Produces single sided print layout (the default if you remove oneside)
%
% The BYUPhys class provides the following macros:
%
%   \makepreliminarypages : Makes the preliminary pages
%   \clearemptydoublepage : same as \cleardoublepage but doesn't put page numbers
%                           on blank intervening pages
%   \singlespace          : switch to single spaced lines
%   \doublespace          : switch to double spaced lines
%
% --------------------------- Load Packages ---------------------------------

% The graphicx package allows the inclusion of figures.  Plain LaTeX and
% pdfLaTeX handle graphics differently. The following code checks which one
% you are compiling with, and switches the graphicx package options accordingly.
\usepackage{ifpdf}
\ifpdf
  \usepackage[pdftex]{graphicx}
\else
  \usepackage[dvips]{graphicx}
\fi

% The fancyhdr package allows you to easily customize the page header.
% The settings below produce a nice, well separated header.
\usepackage{fancyhdr}
  \fancyhead{}
  \fancyhead[LO]{\slshape \rightmark}
  \fancyhead[RO,LE]{\textbf{\thepage}}
  \fancyhead[RE]{\slshape \leftmark}
  \fancyfoot{}
  \pagestyle{fancy}
  \renewcommand{\chaptermark}[1]{\markboth{\chaptername \ \thechapter \ \ #1}{}}
  \renewcommand{\sectionmark}[1]{\markright{\thesection \ \ #1}}

% The caption package allows us to change the formatting of figure captions.
% The commands here change to the suggested caption format: single spaced and a bold tag
\usepackage[margin=0.3in,labelfont=bf,labelsep=none]{caption}
 \DeclareCaptionFormat{suggested}{\singlespace#1#2 #3\par\doublespace}
 \captionsetup{format=suggested}

% The cite package cleans up the way citations are handled.  For example, it
% changes the citation [1,2,3,6,7,8,9,10,11] into [1-3,6-11].  If your advisor
% wants superscript citations, use the overcite package instead of the cite package.
\usepackage{cite}
\usepackage{subcaption}

% The makeidx package makes your index for you.  To make an index entry,
% go to the place in the book that should be referenced and type
%  \index{key}
% An index entry labeled "key" (or whatever you type) will then
% be included and point to the correct page.
\usepackage{makeidx}
\makeindex

% The url package allows for the nice typesetting of URLs.  Since URLs are often
% long with no spaces, they mess up line wrapping.  The command \url{http://www.physics.byu.edu}
% allows LaTeX to break the url across lines at appropriate places: e.g. http://www.
% physics.byu.edu.  This is helpful if you reference web pages.
\usepackage{url}
\urlstyle{rm}

% If you have a lot of equations, you might be interested in the amstex package.
% It defines a number of environments and macros that are helpful for mathematics.
% We don't do much math in this example, so we haven't used amstex here.
\usepackage{amsmath}
% The hyperref package provides automatic linking and bookmarking for the table
% of contents, index, equation references, and figure references.  It must be
% included for the BYU Physics class to make a properly functioning electronic
% thesis.  It should be the last package loaded if possible.
%
% To include a link in your pdf use \href{URL}{Text to be displayed}.  If your
% display text is the URL, you probably should use the \url{} command discussed
% above.
%
% To add a bookmark in the pdf you can use \pdfbookmark.  You can look up its usage
% in the hyperref package documentation
\usepackage[bookmarksnumbered,pdfpagelabels=true,plainpages=false,colorlinks=true,
            linkcolor=black,citecolor=black,urlcolor=blue]{hyperref}

% ------------------------- Fill in these fields for the preliminary pages ----------------------------
%
% For Senior and honors this is the year and month that you submit the thesis
% For Masters and PhD, this is your graduation date
  \Year{2021}
  \Month{December}
  \Author{Lincoln B. Houghton}

% If you have a long title, split it between two lines. The \TitleBottom field defines the second line
% A two line title should be an "inverted pyramid" with the top line longer than the bottom.
  \TitleTop{Modeling of Silver-Platinum Alloy Configuration Energies}
%  \TitleBottom{}

% Your research advisor
  \Advisor{Lance Nelson}

% Your committee members beyond your advisor
  \MemberA{Lord Richard Datwyler}
  %\MemberB{Summer Houghton}
  \ThesisCoordinator{Evan Hansen}

% The representative of the department who will approve your thesis (usually the chair)
  \DepRep{R. Todd Lines}

% The title of the department representative
  \DepRepTitle{Department Chair}


% The text of your abstract
  \Abstract{\par The primary means of discovering novel, high-performing materials uses computer calculations and simulations to construct a material's phase diagram and determine which phases are thermodynamically stable. These calculations are computationally expensive and and produce a bottleneck in the discovery process. The approach discussed here uses a small number of these calculations for an Ag-Pt compound to train a simple pair-interaction model to predict the remainder of the needed data. Using a spread of training parameters, several models are constructed and evaluated. The accuracy of these models is visualized using a series of heatmaps. Some of the models prove useful and can be used in future work to construct an Ag-Pt phase diagram. In particular, the best model constructed maintained an average error of 1.77eV, exceeding all expectations.}

% Acknowledge those who helped and supported you
  \Acknowledgments{I would like to express my gratitude to my research advisor, Lance Nelson, for guidance and encouragement throughout my research. I am also incredibly grateful to the other BYU-I physics faculty that have taught and shaped me throughout my undergraduate degree. A special thanks to my wife for her perpetual support and love. }

\fussy

\begin{document}

 % Start page counting in roman numerals
 \frontmatter

 % This command makes the formal preliminary pages.
 % You can comment it out during the drafting process if you want to save paper.
 \makepreliminarypages

 \singlespace

 % Make the table of contents.
 \tableofcontents
 \clearemptydoublepage

 % Make the list of figures
 \listoffigures
 \clearemptydoublepage

 \doublespace

 % Start regular page counting at page 1
 \mainmatter

 %\include{Chapter1}
 %\include{Chapter2}
 %\include{Chapter3}
 %\include{orbitchapter}
 
 \section{Introduction}\label{Sect:intro}
\par The discovery of novel, high-performing materials is the main driver for technological and industrial advances. The discovery process presents significant challenges; a material with suitable properties which is also thermodynamically stable must be identified. Rather than manufacturing every conceivable material in a laboratory, a more efficient approach involves using computer simulations and calculations to guide the metallurgist.  The density functional theory (DFT) is a well-known quantum-mechanical tool for calculating formation energies, a critical quantity for determining a material's thermodynamic stability.  These formation energies could then be used to construct the material's phase diagram and determine which phases are thermodynamically stable. 
\par Although accurate, DFT calculations are computationally costly, making ground state searches (involving hundreds of thousands of calculations) and thermodynamic simulations computationally prohibitive. Another approach involves using a small set of accurate DFT data (hundreds of calculations) to construct a model which can calculate much faster.  Such a model would help alleviate the stress on the main bottleneck to finding novelty alloys: computational power. 
\par In this paper the process of building a material model will be explained, starting from a simple toy model and later applied to a real DFT data set for Ag-Pt.  Sections \ref{Sect:background} and \ref{Sect:modelPrep} will cover the math and modeling concepts required to understand the actual model construction in Section \ref{Sect:procedure} and the analysis of its results in Section \ref{Sect:results}. 
 \chapter{Background}\label{Sect:background}
\section{Linear Algebra}\label{Sect:linearAlgebra}
A basic knowledge of linear algebra, data analysis, and signal processing is required to understand how to build simple yet useful models. Samples from an unknown function are presented in fig. \ref{fig:func1Samples}. Building a function that matches the data will now be investigated.

\begin{figure}%[h]
\centering
\includegraphics[scale = 0.6]{Figures/func1Samples}
\caption{An unknown function sampled by 6 data points.
\label{fig:func1Samples}} 
\end{figure}

\par It will be assumed that the function can be expanded using a basis, and in this case, a power series:

\begin{align}
f(x) &= \sum_{n=0}^\infty x^n b_n
	\label{eq:powerSum}\\ 
&= x^0b_0 + x^1b_1 + x^2b_2 + \ldots
	\label{eq:powerSeries}
\end{align}
with the coefficients $b_n$ to be determined via the fitting process. Evaluating eq. \ref{eq:powerSum} at the observed data points in fig. \ref{fig:func1Samples} produces a system of equations:

\begin{align}
f(0) &= b_0 \nonumber \\
f(2) &= b_0 + 2 b_1 + 4 b_2 + \dots \nonumber \\
f(4) &= b_0 + 4 b_1 + 16 b_2 + \dots \nonumber \\
f(6) &= b_0 + 6 b_1 + 36 b_2 + \dots \nonumber \\
f(8) &= b_0 + 8 b_1 + 64 b_2 + \dots \nonumber \\
f(10) &= b_0 + 10 b_1 + 100 b_2 + \dots\\
\end{align}

This system of equations can be expressed in the following matrix form:

\begin{equation} \label{eq:LinAlgSubscript}
\begin{bmatrix}
1 & x_1 & x_1^2 & x_1^3 & x_1^4 \\
1 & x_2 & x_2^2 & x_2^3 & x_2^4 \\
1 & x_3 & x_3^2 & x_3^3 & x_3^4 \\
 & & \vdots & &
\end{bmatrix}
\begin{bmatrix}
b_0 \\
b_1 \\
b_2 \\
b_3 \\
b_4 
\end{bmatrix}
=
\begin{bmatrix}
f(x_1) \\ 
f(x_2) \\
f(x_3) \\ 
\vdots
\end{bmatrix}.
\end{equation}

For notation purposes, this matrix equation can be written as 
\begin{equation}\label{eq:aby}
\mathbf{A}\vec{b}=\vec{y}.
\end{equation}

And finally, $\mathbf{A}$ and $\vec{y}$ in eq. \ref{eq:LinAlgSubscript} can be populated with their true values,

\begin{equation} \label{eq:realValues}
\begin{bmatrix}
1 & 0 & 0^2 & 0^3 & 0^4 \\
1 & 2 & 2^2 & 2^3 & 2^4 \\
1 & 4 & 4^2 & 4^3 & 4^4 \\
1 & 6 & 6^2 & 6^3 & 6^4 \\
1 & 8 & 8^2 & 8^3 & 8^4 \\
1 & 10 & 10^2 & 10^3 & 10^4
\end{bmatrix}
\begin{bmatrix}
b_0 \\
b_1 \\
b_2 \\
b_3 \\
b_4 
\end{bmatrix}
=
\begin{bmatrix}
f(0) \\ 
f(2) \\
f(4) \\ 
f(6) \\
f(8) \\
f(10)
\end{bmatrix}.
\end{equation}


The shape of the matrix $\mathbf{A}$ determines the type of solution that can be found.  If there are fewer equations (rows) than unknowns (columns), the system is underdetermined and if it has more rows than columns it is overdetermined.  


\subsection{Underdetermined systems}
Because the columns of an underdetermined system are linearly dependent, there are an infinite number of vectors $\vec{b}$ that can solve the system.  To arrive at a single vector, the solution must be constrained using some chosen criteria.  A commonly-used criteria is to minimize the $\ell_2$-norm of the solution vector. This $\ell_2$-norm, sometimes referred to as the Euclidean norm, is defined as $||\vec{x}||_2=\sqrt{x_1^2+\ldots+x_n^2}$.

\par Solving an underdetermined system is done by using singular value decomposition (SVD) to simplify the calculation of $\mathbf{A}^{-1}$. SVD allows $\mathbf{A}$ to be factored into three elementary matrices. To solve $\mathbf{A}\vec{b}=\vec{y}$, first the eigenvalues and their corresponding orthonormal eigenvectors of $\mathbf{A}^T\mathbf{A}$ must be found. Matrices $U$ and $V$ contain these eigenvectors as their columns and rows, respectively. $\Sigma$ is a diagonal matrix containing the singular values (the square roots of each non-zero eigenvalue), $\sigma_k$, matching the shape of $\mathbf{A}$. The final result is the complete SVD of $\mathbf{A}$ where $\vec{b}$ can be solved by,
\begin{align}
\mathbf{A} &= U\Sigma V^T \nonumber \\
\vec{b} &= (U\Sigma V^T)^T\vec{y} \nonumber \\
\vec{b} &= V^{T^T}\Sigma^TU^T\vec{y} \nonumber \\
\vec{b} &= V\Sigma^TU^T\vec{y}.\nonumber 
\end{align}
where the diagonals of $\Sigma^T$ are $\frac{1}{\sigma_k}$.

\par The solution vector $\vec{b}$ is the well-known least squares solution to the system. Using this solution vector, the model can be used to make predictions of the function for any $x$ value within the range of sample points. To produce a continuous visual of the model's fit, the model can take in $x$ values for every point in the sample range and plot its respective evaluation. The result of this process can be seen in fig. \ref{fig:func1True}.

\subsection{Overdetermined system}
In an overdetermined system, since the column vectors don't span the space that they live in, there may be no vectors $\vec{b}$ that solve the system.  The ``closest" solution vector can be found by solving a slightly modified problem where the vector $\vec{y}$ is projected onto the subspace formed by the columns.  This is eqivalent to the well-known least squares approach and can be written in matrix form as:

\begin{align}
\mathbf{A}\vec{b} &= \vec{y} \label{eq:Aby} \\
%\mathbf{A}^T(\vec{y} - \mathbf{A} \hat{\vec{b}}) &= 0 \nonumber \\
%\mathbf{A}^T\vec{y} &= \mathbf{A}^T\mathbf{A}\hat{\vec{b}} \nonumber \\
\mathbf{A}^T\mathbf{A}\hat{b} &= \mathbf{A}^T\vec{y}\\
\hat{b} &= (\mathbf{A}^T\mathbf{A})^{-1}\mathbf{A}^T\vec{y}. \nonumber %\label{eq:bSolve}
\end{align}
%where $\mathbf{A}^T$ denotes the transpose of $\mathbf{A}$.

\par Since DFT data points are so costly to generate, the matrices that are typically encountered when building materials models are most-often underdetermined (i.e. more basis functions than data points).


\begin{figure}%[h]
\centering
\includegraphics[scale = 0.6]{Figures/func1True}
\caption{The ``Witch of Agnesi" function with Gaussian noise is shown in blue.  Samples from this function were gathered (circles) and used to construct a model using a simple polynomial basis.  The fit function is depicted in green.
\label{fig:func1True}} 
\end{figure}



\section{Basis Functions and Size of Training Set}\label{Sect:samplesAndFunctions}
\par The quality of the fit will be affected by (1) the size of the training set (number of rows in matrix $\mathbf{A}$) and (2) number of basis functions included in the expansion (columns in matrix $\mathbf{A}$). The convergence of our model with respect to these two parameters must be investigated. In theory there is no limit to the size of the training set or number of basis functions. However, since quantum mechanical (QM) data is costly to generate, the size of the training set has a reasonable upper limit on the order of hundreds. In the example given, an increase in the traning set size or basis functions count will not dramatically affect the computational power required, but becomes a greater concern for models of systems with increasing complexity. 
\par As discussed, the number of basis functions can make a large impact, but another important factor is quality. Though the choice of basis functions in the example above was simple, it can often be a difficult choice. Consider, for example, a Fourier basis. The equivalent of eq. \ref{eq:LinAlgSubscript} in this basis would be

\begin{equation} \label{eq:fourierBasis}
\begin{bmatrix}
\sin(x_1) & \sin(2x_1) \\
\sin(x_2) & \sin(2x_2) & \ldots & \ldots \\
\sin(x_3) & \sin(2x_3) \\
\vdots & & \ddots & & \\
\sin(x_n) & \ldots & & \sin(mx_n)
\end{bmatrix}
\begin{bmatrix}
b_0 \\
b_1 \\
b_2 \\
\vdots \\
b_m 
\end{bmatrix}
=
\begin{bmatrix}
f(x_1) \\ 
f(x_2) \\
f(x_3) \\ 
\vdots \\
f(x_n)
\end{bmatrix},
\end{equation}
where $\mathbf{A}$ is an $n\times m$ matrix.
\par When used in the proper circumstance, this Fourier basis can be an excellent choice for modeling a function, as in fig. \ref{fig:2dFourier}.

\begin{figure}%[h]
\centering
\includegraphics[scale = 0.43]{Figures/2dFourier}
\caption{The various sample points shown were used to construct the function model seen in green. The true function is displayed in blue for comparison. The Fourier basis resulted in an excellent fit. 
\label{fig:2dFourier}} 
\end{figure}

\par When chosing sample points, there are again two variables to consider: number and breadth. As will be seen later, the number of samples can have a significant impact on the construction time and accuracy of a model. The breadth of samples is similarly important. If all samples from fig. \ref{fig:2dFourier} were taken between 0 and 1, the model produced would be a poor fit for the function, as in fig. \ref{fig:poorSamps}. 

\begin{figure}%[h]
\centering
\includegraphics[scale = 0.6]{Figures/poorSamps}
\caption{Gathering samples that are not uniformly distributed across the function space will produce a model that does not predict well across the space. In this case, the function from fig. \ref{fig:2dFourier} was only sampled from the interval 0 to 1. If the model were evaluated at $x=2.5$, the fit would produce a highly inaccurate estimate when compared to the true function.
\label{fig:poorSamps}} 
\end{figure}

\par By this point it should be obvious to the reader that the decision of number and breadth of sample points as well as quantity and quality of basis functions is critical to the model's performance. With well chosen basis functions but poor breadth of samples, a model's accuracy can be greatly limited, as in fig. \ref{fig:poorSamps}. Fig. \ref{fig:3dFourier} shows the reverse situation, good number and breadth of samples, but poorly chosen basis functions. To contrast the poor fit of fig. \ref{fig:3dFourier}, the power basis in fig. \ref{fig:3dPower} produces an excellent fit with very few sample points.

\begin{figure}
  \centering
  \begin{subfigure}{0.85\textwidth}
    \includegraphics[width=\linewidth]{Figures/3dFourier}
    \caption{Fourier basis. $\mathbf{A}$ is a $50\times7$ matrix.} 
    \label{fig:3dFourier}
  \end{subfigure}%
  %\hspace*{\fill}
  \\
  \begin{subfigure}{0.8\textwidth}
    \includegraphics[width=\linewidth]{Figures/3dPower}
    \caption{Power basis. $\mathbf{A}$ is a $10\times7$ matrix.} 
    \label{fig:3dPower}
  \end{subfigure}%
\caption{A three dimensional function in blue with a constructed model in green. The red points show the sampling of the true function. (a) In this case, a Fourier basis is a poor choice and resulted in an inaccurate model regardless of the number of sampling points. (b) A Power series makes a convincing fit even with significantly fewer sample points.} \label{fig:3dFit}
\end{figure}

 \section{Preliminary Modeling}\label{Sect:modelPrep}

\subsection{Lennard-Jones Potential}\label{Sect:LJPotential}
Before jumping to quantum mechanical data, this math will be tested to construct models on a simplified potential, the Lennard-Jones potential. Though the Lennard-Jones potential is a simplification of reality, it does a excellent job representing real intermolecular forces of attraction and repulsion. The potential is a function of distance between two particles given by

\begin{equation} \label{LJ}
V_{LJ}(r) = 4\varepsilon \bigg[\Big(\frac{\sigma}{r}\Big)^{12} - \Big(\frac{\sigma}{r}\Big)^6\bigg],
\end{equation}
where $\varepsilon$ and $\sigma$ are constants for a given particle interaction. Figure \ref{figLJ} shows the plot of this potential.

\begin{figure}[h]
\includegraphics[scale = 0.4]{Figures/LJPotential}
\caption{The Lennard-Jones potential. A simple yet realistic model of intramolecular forces.
\label{figLJ}} 
\end{figure}

\par It can be recalled that the force from a potential is given by

\begin{equation} \label{forceEq}
F = -\nabla U,
\end{equation}
and thus the force between two particles is zero at the bottom of the potential well. With that location as a reference, distances any greater will produce a force that is attractive and at any lesser distances, the force is intensely repulsive.
\par Calculating the potential between two particles is not difficult, but finding the total potential energy of a system of several particles becomes increasingly computationaly expensive. Because of the simplicity of the Lennard-Jones potential, the computational power required to solve for the system's energy is still relatively small. In preparation for using real, quantum-mechanical data, each system's energy will be treated as expensive to compute. 



\subsection{Constructing a Model}\label{Sect:LJModels}
\par Ensuring a sufficient number and breadth of samples as well as reasonable basis functions becomes difficult as the number of dimensions goes beyond 2 or 3. From studying a variety of potential basis functions, bessel functions of the second kind, $Y_\alpha(x)$, have the potential to be useful.
\par Rewriting Equation \ref{fundLinAlg} in component form yields


\begin{equation}
\begin{bmatrix}
A_{11} & A_{12} & \ldots & A_{1m} \\
A_{21} \\
\vdots & & & \vdots\\
A_{n1} & \ldots & & A_{nm}
\end{bmatrix}
\begin{bmatrix}
b_0 \\
b_1 \\
b_2 \\
\vdots \\
b_m 
\end{bmatrix}
=
\begin{bmatrix}
V_1 \\
V_2 \\
V_3 \\ 
\vdots \\
V_n
\end{bmatrix}.
\label{AMatrix}
\end{equation}

\par Recognizing each row of $A$ as a unique sample, Equation \ref{AMatrix} can be written as a system of linear equations

\begin{align}
A_{11}b_1 + A_{12}b_2 + A_{13}b_3 + ... + A_{1n}b_n &= V_1 \\
A_{21}b_1 + A_{22}b_2 + A_{23}b_3 + ... + A_{2n}b_n &= V_2.
\end{align}

\par Each element of $A$ will be populated with
\begin{align}
A_{11} &= Y_0(\alpha_{01} r_{12}) + Y_0(\alpha_{01} r_{13}) + Y_0(\alpha_{01} r_{23}) + \ldots \\
A_{12} &= Y_0(\alpha_{02} r_{12}) + Y_0(\alpha_{02} r_{13}) + Y_0(\alpha_{02} r_{23}) + \ldots  \\
A_{1n} &= Y_0(\alpha_{0n} r_{12}) + Y_0(\alpha_{0n} r_{13}) + Y_0(\alpha_{0n} r_{23}) + \ldots
\end{align}
where $\alpha_{0n}$ is the $n$-th zero of $J_0$, the zeroth bessel function of the first kind.

\begin{figure}[h]
\includegraphics[scale = 0.3]{Figures/tenParticles}
\caption{A random assortment of 10 particles in a 4x4 square. No two particles are allowed to be within a pre-specified distance of each other. 
\label{tenParticles}} 
\end{figure}

\begin{itemize}
\item Minimum separation distance
\item Number of particles in box
\item Sample/function num
\item Why separation vector norms?
\item 
\end{itemize}




\subsection{Multi-Type Particle Systems}\label{Sect:diatomic}
\par Now that a simple monotomic model has been constructed, it can be adjusted to handle two different types of particles. The different particles can be labeled type-$A$ and type-$B$. This diatomic system will now require \textit{three} different potential equations, one describing each type of interaction. The three unique interactions are type-$A$ interacting with another type-$A$, a type-$A$ interacting wtih a type-$B$, and a type-$B$ with another type-$B$. Because these are arbitrary interactions, they can be defined by choosing reasonable values of $\varepsilon$ and $\sigma$ from Equation \ref{LJ}. 

\begin{align}
V_{AB}(r) &= 4 \bigg[\Big(\frac{1}{r}\Big)^{12} - \Big(\frac{1}{r}\Big)^6\bigg] \label{LJ} \\
V_{AA}(r) &= 4 (0.7) \bigg[\Big(\frac{0.8}{r}\Big)^{12} - \Big(\frac{0.8}{r}\Big)^6\bigg] \label{LJ} \\
V_{BB}(r) &= 4 (0.4) \bigg[\Big(\frac{1.1}{r}\Big)^{12} - \Big(\frac{1.1}{r}\Big)^6\bigg] \label{LJ}
\end{align}
The graph of each potential can be seen in Figure \ref{fig3LJ}. Each of the equilibrium positions and energies is slightly different, but all in the same neighborhood.

\begin{figure}[h]
\includegraphics[scale = 0.4]{Figures/newLJPotential}
\caption{A Lennard-Jones potential for each particle interaction type (AA, AB, and BB). Each interaction potential has a slightly different equilibrium position and energy. 
\label{fig3LJ}} 
\end{figure}

\par To handle this increase in complexity, matrix $A$ will need to contain \textit{three} columns where the previous model had only one. This is again because of the three different interaction types, one column for each. Another obstacle arises in the decision for particle type ratio. If the particle number (10) and box size ($4\times4$) remain constant, how many type-$A$ versus type-$B$ particles should there be? The question is really just a general question specified to this case; what is a sufficient breadth of samples for this model? As previously discussed, this depends greatly upon the desired range of predictions. As will be seen in Chapter \ref{procedureData}, the configurations vary in particle number as well as ratio. Therefore, this model should be trained and tested using a variety of particle ratios. Each training and testing set will thus be given a random number and ratio of particles to be populated in and calculated. 
\par Once a model has been properly constructed and trained, how can its accuracy be easily determined and visualized?


\begin{itemize}
\item Particle type ratio
\item Show model accuracy
\end{itemize}




 \chapter{Procedure} \label{Sect:procedure}
\section{Quantum Mechanical Data}\label{Sect:procedureData}
\par First-principles calculations were performed within the framework of AFLOW,\cite{monster,curtarolo:art13,curtarolo:art49,curtarolo:art53,curtarolo:art51,curtarolo:art56,curtarolo:art58,curtarolo:art64} which employs the \textsc{vasp} software for computing energies.\cite{kresse1993abinitio} Projector-augmented-wave (PAW) potentials were used and exchange-correlation functionals parametrized by Perdew, Burke, and Ernzerhof under the generalized gradient approximation (GGA)\cite{Kresse:1999wc,kresse1996efficiency,Blochl:1994dx}. A dense $k$-mesh scheme was used to perform the numeric integration over the Brillioun zone\cite{monkhorst1976special}. Optimal choices of the unit cells, by standardization of the reciprocal lattice, were adopted to accelerate the convergence of the calculations. \cite{curtarolo:art58,curtarolo:art64}
\par As previously discussed, the process of obtaining this data was computationally expensive, creating a bottleneck in the discovery of novelty alloys. The culmination of this research is to produce a simpler model that can be trained on the given data to reproduce expected configuration energies within a range of reasonable error. Fig. \ref{histEnergy} shows a histogram of all the configuration energies in the data set.

\begin{figure}%[h]
\centering
\includegraphics[scale = 0.45]{Figures/UnitCellEnergies}
\caption{A histogram showing all the configuration energies in the original data file. The majority lie within the $-20$eV to $-60$eV range, with a few outliers around $-150$eV. These outliers will result in configuration energies that are difficult for the models to predict.
\label{histEnergy}} 
\end{figure}

\par Each configuration in the data represents a unique primitive unit cell. A unit cell is the building block of any crystal structure. Each unit cell is an identical copy of every other, with the same shape, size, and contents. A \textit{primitive} unit cell is the smallest possible unit cell which contains only one of each uniquely positioned atoms in the crystal\cite{solidStateBook}. An example of a primitive unit cell configuration can be seen in fig. \ref{primitiveUnitCells}.

\begin{figure}
  \centering
  \begin{subfigure}{0.53\textwidth}
    \includegraphics[width=\linewidth]{Figures/primitiveCell1}
    %\caption{First subfigure} 
    \label{primitiveFirst}
  \end{subfigure}%
  \hspace*{\fill}   % maximize separation between the subfigures
  \begin{subfigure}{0.49\textwidth}
    \includegraphics[width=\linewidth]{Figures/primitiveCell2}
    %\caption{Second subfigure} 
    \label{primitiveSecond}
  \end{subfigure}%
\caption{An example of a primitive unit cell configuration from two different perspectives. The blue point is the single Ag atom while each red point shows the Pt atoms.
\label{primitiveUnitCells}}
\end{figure}

\par The data being used is in a \textit{.txt} file, thus it needs to be parsed into vectors that can be easily manipulated. An example of the raw data can be seen in fig. \ref{system2data}. The number on the second line is the lattice parameter, followed by three lattice vectors in $(i,j,k)$ coordinates. The line following contains two numbers, the number of silver (Ag) and platinum (Pt) atoms. In direct coordinates (in terms of the lattice vectors) the positions of each silver and platinum atom are then given. The last line in this system tells the total potential energy of the unit cell configuration. 

\begin{figure}%[h]
\centering
\includegraphics[scale = 0.47]{Figures/system2}
\caption{An example configuration from the data given in the \textit{.txt} file. The entire file must be read and parsed into vectors so the data can be organized to construct a model.
\label{system2data}} 
\end{figure}

\section{Model Construction}\label{Sect:procedureConstruction}
\par With the file parsed and the important data retrieved, the process of building a model can commence. The potential energy of a single particle is due to its interactions with all its surrounding particles, thus to account for each interaction with nearby particles the unit cell must be propagated outwards in all three dimensions. All atomic pairs can be enumerated by adding multiples of the lattice vectors. Then the relative position of each affecting particle can be determined and the vector separating the particle pair can be calculated. These separation vectors are the information that will be passed into the basis function to construct the model. 
\par Because the effect two particles have on each other drops off as a function of distance (similar to the Lennard-Jones potential), the unit cell does not need to be propogated infinitely in each direction, only out to a radius of reasonable influence, creating an imaginary sphere beyond which all interactions are negligible. It should be clear that the choice of this radius will have a significant impact on the quality of the model. It should also be noted that one arbitrarily chosen radius cannot be applied effectively to each unique system. A simple solution is to make the radius of this sphere of influence equal to the magnitude of the largest lattice vector multiplied by a constant. The effect of this constant on the model's precision can be tested later, but for now will be chosen to be 1.2.
\par The simplification of reality due to this ``sphere of influence" gives a clear upper bound to our unit cell propagation. It is expected that the unit cell will be propagated out further than the radius in each direction, thus encapsulating said sphere. An example of this iterated unit cell can be seen in fig. \ref{iteratedUnitCells}. 

\begin{figure}
  \centering
  \begin{subfigure}{0.53\textwidth}
    \includegraphics[width=\linewidth]{Figures/iteratedUnitCell}
    %\caption{First subfigure} 
    \label{iteratedUnitCellFirst}
  \end{subfigure}%
  %\\
  \hspace*{\fill}   % maximize separation between the subfigures
  \begin{subfigure}{0.5\textwidth}
    \includegraphics[width=\linewidth]{Figures/iteratedUnitCell2}
    %\caption{Second subfigure} 
    \label{iteratedUnitCellSecond}
  \end{subfigure}
\caption{The sphere of influence for a single silver atom is completely encapsulated by the iterations of the unit cell as seen from two different perspectives. Only one particle from each cell is shown.} \label{iteratedUnitCells}
\end{figure}

\par Each unit cell in fig. \ref{iteratedUnitCells} can now be populated with all the contained atoms. Every atom inside the sphere can then be extracted. When these atoms are stored into a new vector, they can be plotted as in fig. \ref{plotAgAtomSphere}. When this process is repeated for every system in the DFT data, the result is a vector containing these ``important positions" for each unique silver and platinum atom in each unit cell. The separation vector norms can be quickly calculated and stored.

\begin{figure}%[h]
\centering
\includegraphics[scale = 0.85]{Figures/plotAgAtomSphere(3,1)}
\caption{Every silver and platinum atom inside the sphere of interest centered on a single silver atom. Due to the cutoff radius, these are the only atoms interacting with the central atom. This data was generated for the first silver atom in the third sample configuration. 
\label{plotAgAtomSphere}} 
\end{figure}

\par With all the pertinent information from the data organized into vectors, construction of the $\mathbf{A}$ matrix can begin. As in the Lennard-Jones example, each row represents a crystal structure and each column corresponds to a unique basis function. And as explained in Chapter \ref{Sect:diatomic}, a single basis function will result in three columns of $\mathbf{A}$ due to each type of interaction. The method of populating $\mathbf{A}$ will be the same as was done earlier in eq. \ref{eq:fillBessel}. With 300 basis functions, the first 300 columns of $\mathbf{A}$ will be calculated using the separation vector norms from Ag-Ag interactions. The following 300 columns from Pt-Pt interactions and the final 300 from Ag-Pt.
\par The size of the training set is limited to 1224, the number of systems given in the data set. Whatever systems are not included in the training set will comprise the holdout set, used for testing the model's accuracy.
%\par The basis functions will again be chosen to be bessel functions of the second kind. This is because of the similarity in potential energy between two atoms as a function of distance and the shape of the graph of said bessel functions. 
\par Once the model is fully constructed, tests can be run to determine the effect of radius, size of training set, and number of basis functions on the precision and accuracy of the model. An increase of these three variables will make significant impacts on the program runtime. As the radius is increased, the unit cell must be propagated out further to encapsulate the sphere of influence and more atom interactions will be considered. For every basis function added, there will be three columns added to the $\mathbf{A}$ matrix, drawing out its construction times. 
\par Generally, as the number of basis functions is increased and the radius of influence extended, the model's predictions will become increasingly reliable. On the other hand, as those factors increase accuracy and precision, they also increase the computational costs. It would be ideal to find a manageable trade off between the program's runtime and reliability. To investigate the quality of the model as a function of cutoff radius, size of training set, and number of basis functions, a sweep can be conducted over a range of values for each and the details of the fit for each combination can be recorded. Rather than run this script on an ordinary desktop computer, it was completed on Mary Lou, BYU's supercomputer. 

 \chapter{Results and Analysis}\label{Sect:results}
\par The quality of fit for a single model is given in fig. \ref{outputExample}. For some models, there are predictions that differ significantly, more than 100eV, from their true values, these are configurations that are difficult for the model to predict. In those cases, the number of outlying crystals has been counted then excluded from the average error calculation. In this way, the average error can become mostly independent of the model's outliers. When a difficult-to-predict configuration is poorly predicted and thus excluded from the average error, it can reduce the overall average error for a given model. A better trained model may then predict that configuration energy slightly better, within 100eV of its true value, resulting in a larger average error. Therefore the average error and number of large errors must both be considered when assessing the value of the model. 

\begin{figure}%[h]
\centering
\includegraphics[scale = 0.6]{Figures/outputExample}
\caption{An example of the data in the output file generated on the supercomputer. The text gives the size of the training set, number of basis functions, and the cutoff radius used to generate the model. It also tells the size of the holdout set, the average error, and the number of errors larger than 100eV.
\label{outputExample}} 
\end{figure}

\par To understand a given model's accuracy and precision visually, each prediction energy can be plotted against the actual energy for each configuration. The model referenced in fig. \ref{outputExample} can be seen in this format in fig. \ref{accuracyPlot}.

\begin{figure}%[h]
\centering
\includegraphics[scale = 0.5]{Figures/accuracyPlot}
\caption{Predicted energies versus the actual energies for each configuration in the holdout set. This model is the same as the one shown in fig. \ref{outputExample}. The prediction energies seem to form an adequate match to their true values. When the simplicity of this model is contrasted with the high levels of complexity it is attempting to capture, the accuracy of this fit is impressive.
\label{accuracyPlot}} 
\end{figure}

\par This method of visual analysis in fig. \ref{accuracyPlot} works very well for observing one model at a time but interpreting each of the 270 unique models this way would be difficult and time consuming. Preferably, several data sets could be interpreted at one time.
\par One of many possible solutions is to use a series of heatmaps showing the average error of each model with it's respective parameters. But as discussed above, the average error does not tell the whole story; the number of large errors must also be accounted for. Because the size of the holdout set changes from model to model, it would be more useful to display the percentage of large errors rather than just the number of large errros. Fig. \ref{aveErrorHeatmaps} shows five heatmaps, one for each size of training set. The color in each square represents the average error for a specific model. The scale for each subplot is set to be identical to make its analysis easier. A similar cluster of plots can be seen in fig. \ref{perErrorHeatmaps}, showing the number of large errors for each model. 

\begin{figure*}
  \centering
  \begin{subfigure}{0.5\textwidth}
    \includegraphics[width=\linewidth]{Figures/aveErrors2}
    \caption{} 
    \label{aveErrors2}
  \end{subfigure}%
  \hspace*{\fill}   % maximize separation between the subfigures
  \begin{subfigure}{0.5\textwidth}
    \includegraphics[width=\linewidth]{Figures/aveErrors4}
    \caption{} 
    \label{aveErrors4}
  \end{subfigure}%
    %\hspace*{\fill}   % maximize separation between the subfigures
    \\
  \begin{subfigure}{0.5\textwidth}
    \includegraphics[width=\linewidth]{Figures/aveErrors6}
    \caption{} 
    \label{aveErrors6}
  \end{subfigure}%
    \hspace*{\fill}   % maximize separation between the subfigures
  \begin{subfigure}{0.5\textwidth}
    \includegraphics[width=\linewidth]{Figures/aveErrors8}
    \caption{} 
    \label{aveErrors8}
  \end{subfigure}%
    %\hspace*{\fill}   % maximize separation between the subfigures
    \\
  \begin{subfigure}{0.5\textwidth}
    \includegraphics[width=\linewidth]{Figures/aveErrors10}
    \caption{} 
    \label{aveErrors10}
  \end{subfigure}%
\caption{Heatmaps showing the average error of each model produced. The scale for each is fixed from 0eV to 10eV. The general trend is that each plot is more blue on the left and red on the right. This is because many of the less sophisticated models predict the more difficult configurations poorly, causing them to be excluded from the average error. As the sophistication of the model increases, it will predict these configurations with greater accuracy and thus raise the average error by a noticeable amount. This can be verified by comparing these heatmaps with those in fig. \ref{perErrorHeatmaps}.
\label{aveErrorHeatmaps}}
\end{figure*}


\begin{figure*}
  \centering
  \begin{subfigure}{0.5\textwidth}
    \includegraphics[width=\linewidth]{Figures/perErrors2}
    \caption{} 
    \label{perErrors2}
  \end{subfigure}%
  \hspace*{\fill}   % maximize separation between the subfigures
  \begin{subfigure}{0.5\textwidth}
    \includegraphics[width=\linewidth]{Figures/perErrors4}
    \caption{} 
    \label{perErrors4}
  \end{subfigure}%
    %\hspace*{\fill}   % maximize separation between the subfigures
    \\
  \begin{subfigure}{0.5\textwidth}
    \includegraphics[width=\linewidth]{Figures/perErrors6}
    \caption{} 
    \label{perErrors6}
  \end{subfigure}%
    \hspace*{\fill}   % maximize separation between the subfigures
  \begin{subfigure}{0.5\textwidth}
    \includegraphics[width=\linewidth]{Figures/perErrors8}
    \caption{} 
    \label{perErrors8}
  \end{subfigure}%
    %\hspace*{\fill}   % maximize separation between the subfigures
    \\
  \begin{subfigure}{0.5\textwidth}
    \includegraphics[width=\linewidth]{Figures/perErrors10}
    \caption{} 
    \label{perErrors10}
  \end{subfigure}%
\caption{Heatmaps showing the percentage of large errors of each model produced. The scale for each is fixed from 0-0.05, or 0-5\%. A model with a percent error of 0.01 would mean that 1 out of every 100 predictions has an error greater than 100eV. The percent error is preferred over the number of large errors because it takes into account the fact that as the size of the training set increases, the size of the holdout set decreses. Thus a model with a large training set will naturally have a smaller number of large errors, but not necessarily a smaller percentage of large errors.
\label{perErrorHeatmaps}}
\end{figure*}


\par The true accuracy and precision of each model can only be realized when comparing the results from fig. \ref{aveErrorHeatmaps} \textit{and} fig. \ref{perErrorHeatmaps}. One without the other does not show the full picture. For example, the lower left corner of fig. \ref{aveErrors2} appears to be surprisingly accurate, but when compared with the corresponding squares in fig. \ref{perErrors2}, a considerable percent of large errors can be seen. These particular models are not of interest. The models that \textit{are} of interest will be the squares from corresponding subplots in fig. \ref{aveErrorHeatmaps} and fig. \ref{perErrorHeatmaps} that are both blue or blue-ish. 
%\par fig. \ref{numErrors2} includes a large red clump in the bottom right of the plot. This red patch then appears to move upward in each ensuing figure. \ldots \ldots \ldots
\par The most successful model was trained on 1000 configurations, used 40 basis functions, and had an effective radius of 1.0. This model produced 0 large errors and had an average error of only 1.77eV. As expected, the second and third best models had very similar input parameters.

 \chapter{Conclusion}\label{Sect:conclusion}
\par The models produced were satisfactorily successful. Each provided great insight into which parameters build useful models. Though some models are useful, none are without errors. Using the ``best" model, as mentioned in Chapter \ref{Sect:results}, new data on silver-platinum crystal configurations can be produced.


\section{Future Work}\label{Sect:futureWork}
\par Using a small set of accurate DFT data, the code written to produce the model in Chapter \ref{Sect:procedureConstruction} can be easily adapted to construct models for compounds other than Ag-Pt. Such research could be used to determine which compounds can be accurately predicted using a simple pair-interaction model. A successful model can quickly predict formation energies for a variety of crystal configurations. These formation energies could then be used to construct the material’s phase diagram and determine which phases are thermodynamically stable.
\par Because the models produced in this paper used only pair-interactions, their effectiveness is limited by their simplicity. Implementation of three-body interactions has the potential to greatly increase the accuracy of these models.
%\par Citations yet to be included: \cite{_AFLOW} \cite{Kohn1965} \cite{curt2005scienceandtech} \cite{curtarolo2003predicting} \cite{Kresse:1999wc} \cite{kresse1993abinitio} \cite{Blochl:1994dx} \cite{kresse1996efficiency} \cite{monkhorst1976special} \cite{sanchez1984generalized} \cite{laks1992efficient} \cite{lerch2009uncle} \cite{cockayne2010building} \cite{nelson2013compressive}
 

 \appendix


 \bibliographystyle{phBYU}
 \bibliography{../refs,../refs2}

 %\include{AppendixA}
 %\include{AppendixB}

% \phantomsection \addcontentsline{toc}{chapter}{Index}
  \renewcommand{\baselinestretch}{1} \small \normalsize
  \printindex


\end{document}
