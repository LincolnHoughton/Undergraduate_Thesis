\section{Introduction}\label{Sect:intro}
\par The discovery of novel, high-performing materials is the main driver for technological and industrial advances. The discovery process presents significant challenges; a material with suitable properties which is also thermodynamically stable must be identified. Rather than manufacturing every conceivable material in a laboratory, a more efficient approach involves using computer simulations and calculations to guide the metallurgist.  The density functional theory (DFT) is a well-known quantum-mechanical tool for calculating formation energies, a critical quantity for determining a material's thermodynamic stability.  These formation energies could then be used to construct the material's phase diagram and determine which phases are thermodynamically stable. 
\par Although accurate, DFT calculations are computationally costly, making ground state searches (involving hundreds of thousands of calculations) and thermodynamic simulations computationally prohibitive. Another approach involves using a small set of accurate DFT data (hundreds of calculations) to construct a model which can calculate much faster.  Such a model would help alleviate the stress on the main bottleneck to finding novelty alloys: computational power. 
\par In this paper the process of building a material model will be explained, starting from a simple toy model and later applied to a real DFT data set for Ag-Pt.  Sections \ref{Sect:background} and \ref{Sect:modelPrep} will cover the math and modeling concepts required to understand the actual model construction in Section \ref{Sect:procedure} and the analysis of its results in Section \ref{Sect:results}. 