\section{Background}\label{Sect:background}
\subsection{Linear Algebra}\label{Sect:linearAlgebra}
A knowledge of linear algebra, data analysis, and signal processing is required to understand how to build simple yet useful models. Samples from a secret function are presented in Figure \ref{fig:func1Samples}. Building a function that matches the data will now be investigated.

\begin{figure}[h]
\includegraphics[scale = 0.4]{Figures/func1Samples}
\caption{An unknown function sampled by 6 data points.
\label{fig:func1Samples}} 
\end{figure}

\par We'll assume that the function can be expanded using a basis, and in this case we'll assume a power series:

\begin{align}
f(x) &= \sum_{n=0}^\infty b_n x^n
	\label{eq:powerSum}\\ 
&= x^0b_0 + x^1b_1 + x^2b_2 + \ldots\ .
	\label{eq:powerSeries}
\end{align}

with the coefficients $b_n$ to be determined via the fitting process.  Evaluating equation \ref{eq:powerSeries} at the observed data points in the figure produces a system of equations:

\begin{align}
f(0) &= b_0\\
f(2) &= b_0 + 2 b_1 + 4 b_2 + \dots\\
f(4) &= b_0 + 4 b_1 + 16 b_2 + \dots\\
f(6) &= b_0 + 6 b_1 + 36 b_2 + \dots\\
f(8) &= b_0 + 8 b_1 + 64 b_2 + \dots\\
f(10) &= b_0 + 10 b_1 + 100 b_2 + \dots\\
\end{align}


This system of equations can be expressed in the following matrix form:

\begin{equation} \label{eq:LinAlgSubscript}
\begin{bmatrix}
1 & x_1 & x_1^2 & x_1^3 & x_1^4 \\
1 & x_2 & x_2^2 & x_2^3 & x_2^4 \\
1 & x_3 & x_3^2 & x_3^3 & x_3^4 \\
 & & \vdots & &
\end{bmatrix}
\begin{bmatrix}
b_0 \\
b_1 \\
b_2 \\
b_3 \\
b_4 
\end{bmatrix}
=
\begin{bmatrix}
f(x_1) \\ 
f(x_2) \\
f(x_3) \\ 
\vdots
\end{bmatrix}.
\end{equation}


%\par Now we can see that each row in the $x$ matrix corresponds with a row in the $f(x)$ vector. Each row represents one of our sample points, thus producing 6 rows. This relationship between rows and individual sample points can be better visualized if subscripts are added to each respective value of $x$,

And finally, $A$ and $\vec{y}$ can be populated with their true values,

\begin{equation} \label{eq:realValues}
\begin{bmatrix}
1 & 0 & 0^2 & 0^3 & 0^4 \\
1 & 2 & 2^2 & 2^3 & 2^4 \\
1 & 4 & 4^2 & 4^3 & 4^4 \\
1 & 6 & 6^2 & 6^3 & 6^4 \\
1 & 8 & 8^2 & 8^3 & 8^4 \\
1 & 10 & 10^2 & 10^3 & 10^4
\end{bmatrix}
\begin{bmatrix}
b_0 \\
b_1 \\
b_2 \\
b_3 \\
b_4 
\end{bmatrix}
=
\begin{bmatrix}
f(0) \\ 
f(2) \\
f(4) \\ 
f(6) \\
f(8) \\
f(10)
\end{bmatrix}
\end{equation}


The shape of the matrix $\mathbf{A}$ determines the type of solution that can be found.  If there are fewer equations (rows) than unknowns (columns), the system is underdetermined and if it has more rows than columns it is overdetermined.  
\subsubsection{Underdetermined systems}
Because the columns of an underdetermined system are dependent, there are an infinite number of vectors $\vec{b}$ that can solve the system.  To arrive at a single vector, the solution must be constrained using some chosen criteria.  A commonly-used criteria is to minimize the $\ell_2$-norm of the solution vector. 

\par Solving an underdetermined system is done by using singular value decomposition (SVD)\cite{linAlg-book}. To solve $A\vec{b}=\vec{y}$, first find the eigenvalues and their corresponding orthonormal eigenvectors of $A^TA$. $V$ is the matrix containing these eigenvectors and $\Sigma$ contains the singular values, the square roots of each non-zero eigenvalue, on its diagonal and $0$'s elsewhere, matching the shape of matrix $A$. Each column of the matrix $U$ can be constructed by $u_k=\frac{1}{\sigma_k}A\vec{v_k}$ where $\sigma_k$ are the singular values and $\vec{v_k}$ are the columns of $V$. The final result is the complete SVD of $A$ and the solving of $\vec{b}$ from Equation \ref{eq:bSolve} by

\begin{align}
A & =U\Sigma V^T \\
\vec{b} &= (U\Sigma V^T)^T\vec{y} \\
\vec{b} &= V^{T^T}\Sigma^TU^T\vec{y} \\
\vec{b} &= V\Sigma^TU^T\vec{y}.
\end{align}

The solution vector $\vec{b}$ is the well-known least squares solution to the system. Using this solution vector, the model can be used to make predictions of the function for any x value within the range of sample points. To produce a continuous visual of the model's fit, the model can take in x values for every point in the sample range and plot its respective evaluation. The result of this process can be seen in Figure \ref{fig:func1True}.

\subsubsection{Overdetermined system}
In an overdetermined system, since the column vectors don't span the space that they live in, there may be no vectors $\vec{b}$ that solve the system.  The "closest" solution vector can be found by solving a slightly modified problem where the vector $\vec{y}$ is projected onto the subspace formed by the columns.  This is eqivalent to the well-known least squares approach and can be written in matrix form as:

\begin{align}
A\vec{b} &= \vec{y} \\
(A^TA)\vec{b} &= A^T\vec{y} \\
\vec{b} &= (A^TA)^{-1}A^T\vec{y}, \label{eq:bSolve}
\end{align}
where $A^T$ denotes the transpose of $A$.

 %If the $A$ matrix were square and invertible, solving for $\vec{b}$ would be as simple as $\vec{b} = A^{-1}\vec{y}$. Because there are more equations than unknowns, this is an \textit{over-determined} system; the alternative being an \textit{under-determined} system, one with more unknowns than equations.
%\par There will be either no solutions, or an infinite number of solutions. In the case of infinite solutions, the solution needs to be constrained. If there are no solutions, as is often the case with an underdetermined system, a least-squares approximation can be calculated. In either case, the solutions to the linear systems become non-unique. This means there is not a single set of values for $\vec{b}$ that can solve Equation \ref{eq:fundLinAlg}.
%\par In the example given, calculating one sample is quite easy, but in modeling more complicated systems, each sample becomes computationaly expensive. It will later be seen that the systems used will generally be underdetermined.
%\par Solving an overdetermined system for the vector of coefficients can be done by 

Since DFT data points are so costly to generate, the matrices that are typically encountered when building materials models are most-often underdetermined (i.e. more basis functions than data points).


\begin{figure}[h]
\includegraphics[scale = 0.4]{Figures/func1True}
\caption{The true function is shown by the blue line and the points are the locations of our samples. The green line shows the function created by our model to predict the true function.
\label{fig:func1True}} 
\end{figure}



\subsection{Basis Functions and Sample Size}\label{Sect:samplesAndFunctions}
\par As seen in the previous section, and in particular Equations \ref{eq:LinAlgSubscript} and \ref{eq:realValues}, each row of matrix $A$ and vector $\vec{y}$ represents a single sampling and each column is a unique basis function. It should be understood that as either of these two variables increases, thus increasing the length or width of $A$, so should the accuracy and precision of the model increase. In theory, there is no limit to the number of samples or basis vectors that could be used to construct a model, but in reality, there is a balance between the former variables and computational capabilities. In the example given, an increase in the number of samples or basis functions will not dramatically affect the computational power required, but becomes a greater concern for models of systems with increasing complexity. 
\par As discussed, the quantity of basis functions can make a large impact, but another important factor is quality. Though the choice of basis functions in the example above was simple, it can often be a difficult choice. Consider, for example, a Fourier basis. The equivalent of Equation \ref{eq:LinAlgSubscript} in this basis would be

\begin{equation} \label{eq:fourierBasis}
\begin{bmatrix}
\sin(x_1) & \sin(2x_1) \\
\sin(x_2) & \sin(2x_2) & \ldots & \ldots \\
\sin(x_3) & \sin(2x_3) \\
\vdots & & \ddots & & \\
\sin(x_n) & \ldots & & \sin(mx_n)
\end{bmatrix}
\begin{bmatrix}
b_0 \\
b_1 \\
b_2 \\
\vdots \\
b_m 
\end{bmatrix}
=
\begin{bmatrix}
f(x_1) \\ 
f(x_2) \\
f(x_3) \\ 
\vdots \\
f(x_n)
\end{bmatrix},
\end{equation}
for an $n\times m$ matrix.
\par When used in the proper circumstance, this Fourier basis can be an excellent choice for modeling a function, as in Figure \ref{fig:2dFourier}.

\begin{figure}[h]
\includegraphics[scale = 0.27]{Figures/2dFourier}
\caption{The sample points shown were used to build a model of the function in green, and the true function is displayed in blue. As can be seen, a Fourier basis resulted in an excellent fit. 
\label{fig:2dFourier}} 
\end{figure}

\par When chosing sample points, there are again two things to consider: number and breadth. As will be seen later, the number of samples can have a significant impact on the time it takes to construct a model and the accuracy of said model. The breadth of samples is similarly important. If all samples from Figure \ref{fig:2dFourier} were taken between 0 and 1, the model produced would poorly estimate $f(2.5)$, as in Figure \ref{fig:poorSamps}. 

\begin{figure}[h]
\includegraphics[scale = 0.4]{Figures/poorSamps}
\caption{A model with an insufficient breadth of samples will naturally produce a poor model. In this case, the secret function from Figure \ref{fig:2dFourier} was only sampled from the interval 0 to 1. If the model were evaluated at $x=2.5$, the fit would produce a highly inaccurate result when compared to the true function.
\label{fig:poorSamps}} 
\end{figure}

\par By this point it should be obvious to the reader that the decision of number and breadth of sample points as well as quantity and quality of basis functions is critical to the model's performance. With well chosen basis functions but poor breadth of samples, a model's accuracy can be greatly limited, as in Figure \ref{fig:poorSamps}. Figure \ref{fig:3dFourier} shows the reverse situation, good number and breadth of samples, but poorly chosen basis functions. 

%\begin{figure}[h]
%\includegraphics[scale = 0.42]{Figures/3dFourier}
%\caption{A three dimensional function in blue with a constructed model in green. The red points show the sampling of the true function. In this case, a Fourier basis is a poor choice and resulted in an inaccurate model of the true function.
%\label{3dFourier}} 
%\end{figure}

\begin{figure}
  \begin{subfigure}{0.35\textwidth}
    \includegraphics[width=\linewidth]{Figures/3dFourier}
    \caption{Fourier basis. $A$ is a $50\times7$ matrix.} 
    \label{fig:3dFourier}
  \end{subfigure}%
  \\
  \begin{subfigure}{0.35\textwidth}
    \includegraphics[width=\linewidth]{Figures/3dPower}
    \caption{Power basis. $A$ is a $10\times7$ matrix.} 
    \label{fig:3dPower}
  \end{subfigure}%
\caption{A three dimensional function in blue with a constructed model in green. The red points show the sampling of the true function. (a) In this case, a Fourier basis is a poor choice and resulted in an inaccurate model of the true function. (b) A Power series makes a convincingly better fit even with significantly fewer sample points.} \label{fig:3dFit}
\end{figure}


\subsection{Julia}\label{Sect:julia}
The figures provided throughout this paper were produced using the Julia programming language. Though encouraged to use Python for the majority of my formal education, Julia is a relatively new and fast growing language in terms of popularity. 
